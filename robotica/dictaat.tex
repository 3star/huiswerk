\documentclass[a4paper,11pt]{report}
% Hier hebben we de preamble, alle document settings moeten hier:
\usepackage{graphicx}
\usepackage{url}
\usepackage{appendix}
\usepackage[titles]{tocloft}
\usepackage[dutch]{babel}
\usepackage{listings}
\usepackage{makeidx}
\usepackage{float}
\usepackage[hypertexnames=false]{hyperref}

% Custom LaTeX commands:
% ~ == {\raise.17ex\hbox{$\scriptstyle\sim$}}
\newcommand{\customtilde}{\raise.17ex\hbox{$\scriptstyle\sim$}}
% Meta info pdf:
\hypersetup{
%bookmarks=true, % show bookmarks bar?
unicode=false, % non-Latin characters in Acrobat’s bookmarks
pdftoolbar=true, % show Acrobat’s toolbar?
pdfmenubar=false, % show Acrobat’s menu?
pdffitwindow=false, % window fit to page when opened
%pdfstartview={FitH}, % fits the width of the page to the window
pdftitle={Robotica}, % title
pdfauthor={Paul Sohier Sebastiaan Polderman}, % author
pdfsubject={Robotica}, % subject of the document
pdfcreator={make}, % creator of the document
pdfproducer={make}, % producer of the document
pdfkeywords={Linux} {Basis}, % list of keywords
pdfnewwindow=true, % links in new window
colorlinks=false, % false: boxed links; true: colored links
linkcolor=black, % color of internal links
citecolor=green, % color of links to bibliography
filecolor=magenta, % color of file links
urlcolor=cyan % color of external links
}

% Paragrafen hebben een witregel ertussen, en geen indent tab:
\setlength{\parindent}{0.0in}
\setlength{\parskip}{0.1in}
% Onderstaande is voor de dots tussen chapter title + blz. 
\makeatletter
\renewcommand*\l@chapter[2]{%
  \ifnum \c@tocdepth >\m@ne
    \addpenalty{-\@highpenalty}%
    \vskip 1.0em \@plus\p@
    \setlength\@tempdima{1.5em}%
    \begingroup
      \parindent \z@ \rightskip \@pnumwidth
      \parfillskip -\@pnumwidth
      \leavevmode \bfseries
      \advance\leftskip\@tempdima
      \hskip -\leftskip
      #1\nobreak\normalfont\leaders\hbox{$\m@th
        \mkern \@dotsep mu\hbox{.}\mkern \@dotsep
        mu$}\hfill\nobreak\hb@xt@\@pnumwidth{\hss #2}\par
      \penalty\@highpenalty
    \endgroup
  \fi}
\makeatother

% End of title + blz.
% Table of content depth van 4, dus tm paragraph
\setcounter{tocdepth}{4}
%\renewcommand{\baselinestretch}{1.5} 1.5 regelafstand 

%Pas listings aan zodat ze duidelijker zijn
\lstset{ %
  language=bash,                % choose the language of the code
  basicstyle=\footnotesize,       % the size of the fonts that are used for the code
  numbers=left,                   % where to put the line-numbers
  numberstyle=\footnotesize,      % the size of the fonts that are used for the line-numbers
  numbersep=5pt,                  % how far the line-numbers are from the code
  showspaces=false,               % show spaces adding particular underscores
  showstringspaces=false,         % underline spaces within strings
  showtabs=false,                 % show tabs within strings adding particular underscores
  frame=lr,	                % adds left and right lines
  tabsize=2,	                % sets default tabsize to 2 spaces
  captionpos=b,                   % sets the caption-position to bottom
  breaklines=true,                % sets automatic line breaking
  breakatwhitespace=false,        % sets if automatic breaks should only happen at whitespace
%  escapeinside={\%*}{*)},         % if you want to add a comment within your code
  morekeywords={*,...}            % if you want to add more keywords to the set
}
%hyperref aanpassingen
\hypersetup{pdfborder=0 0 0}
% We gebruiken de index
\makeindex

% Einde preamble, begin document; 
\begin{document}
% Title page
\title{
  Robotica eindverslag
}
\author{
  Sebastiaan Polderman\\
  nummer
  \and
  Paul Sohier\\
  0806122
}
\date{\today}
% Print de title
\maketitle

% Abstract (+ in toc)?:
%\begin{abstract}\centering

%\end{abstract}

% De table of contents + toc in toc:
\tableofcontents
\addcontentsline{toc}{chapter}{\numberline{}Inhoudsopgave}

% HACK: Page number fixen. Dit is voor makeidx + hyperref
% 15 is de pagina met hoofdstuk 01, dus inleiding erop. 
% Dit zorgt ervoor dat alle
%\setcounter{page}{3}
% Nu kunnen we de losse hoofdstukken gaan includen. 
% Includen gebeurt met basename, dus zonder .tex
\chapter{Inleiding}
Dit document gaat over het eindresultaat van Project 7 en 8. Wij hebben Project 7 en 8 samengevoegd tot \'{e}\'{e}n groot project waarbij wij \'{e}\'{e}n einddoel hadden. Dit project is hierbij weer verdeeld in diverse kleinere onderdelen. 

In dit document gaan we kort uitleggen wat ons project precies inhield en welke we problemen we hierbij tegenkwamen. Door deze problemen zijn wij helaas niet tot een werkend eindproduct gekomen, maar dit zullen wij verderop in dit document uitgebreid toelichten.

\chapter{De bouw}
Doordat we geen compleet eigen ontwerp gingen maken van Fluffy hebben we eerst helemaal uitgezocht hoe Fluffy precies in elkaar zit. Zonder deze informatie kunnen we hem niet namaken, en doordat we de code niet hebben moeten we er ook voor zorgen dat hij ook goed in elkaar zit zoals de orginele Fluffy. 

Het belangrijkste hiervan zijn de motoren. De motoren worden aangestuurd via het nummer van de motor. Iedere motor heeft een uniek nummer in de robot. Via dit nummer wordt die motor aangestuurd. Wanneer de motor opeens op een andere plek zou zitten als in de orginele Fluffy gaat hij mischien wel lopen in plaats van met zijn staart te kwispellen. En dit willen we uiteraard niet zien gebeuren. 
Naast de motoren moesten we er ook voor zorgen dat alle plastic onderdelen die verder gebruikt zijn op dezelfde manier erop komen. Anders heb je mogelijk verschil in grote van poten, waardoor hij bijvoorbeeld niet meer goed loopt. 

Toen we een complete inventaris hadden van hoe fluffy compleet in elkaar zet konden we beginnen met het bouwen van Fluffy. Het bouwen zelf heeft eigenlijk minder tijd gekost als het complete onderzoek naar Fluffy. We liepen bij de bouw wel met regelmaat tegen allerlei problemen op. Dit bestond voornamelijk uit dat de motoren of plastic onderdelen verkeerd om zaten of het verkeerde onderdeel gebruikt was.

Het hele project lijkt op papier veel minder werk als een normaal project, maar doordat we hem precies moeten namaken heeft dit project meer tijd qua onderzoek gekost als een normaal project. Hiernaast moesten we ook iedere keer controleren of wat we gedaan hadden in dat stukje van Fluffy ook wel klopten met wat er in het orgineel zat. En wanneer dit niet het geval was (Wat zo af en toe wel eens voor kwam), moest dit weer uit elkaar gehaald worden en opnieuw bevestigd. Door dit soort kleine dingen duurt dit project vrij lang.

Omdat Fluffy een niet hele grote hond is hebben we eigenlijk in bijna alle gevallen alleen gewerkt. Dit om ervoor te zorgen dat we elkaar niet in de weg zaten en hierdoor dus nog meer fouten gingen maken bij de bouw. Wel konden we bijvoorbeeld tegelijk werken, maar dan allebei aan een ander onderdeel van Fluffy. Pas op het eind toen we alle lossen onderdelen (Poten, staart, lijf) bij elkaar gingen voegen hebben we met zijn twee\"{e}n aan heel Fluffy gewerkt.


% Einde document
\end{document}
