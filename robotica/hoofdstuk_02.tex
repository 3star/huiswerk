\chapter{De bouw}
Doordat we geen compleet eigen ontwerp gingen maken van Fluffy hebben we eerst helemaal uitgezocht hoe Fluffy precies in elkaar zit. Zonder deze informatie kunnen we hem niet namaken, en doordat we de code niet hebben moeten we er ook voor zorgen dat hij ook goed in elkaar zit zoals de orginele Fluffy. 

Het belangrijkste hiervan zijn de motoren. De motoren worden aangestuurd via het nummer van de motor. Iedere motor heeft een uniek nummer in de robot. Via dit nummer wordt die motor aangestuurd. Wanneer de motor opeens op een andere plek zou zitten als in de orginele Fluffy gaat hij mischien wel lopen in plaats van met zijn staart te kwispellen. En dit willen we uiteraard niet zien gebeuren. 
Naast de motoren moesten we er ook voor zorgen dat alle plastic onderdelen die verder gebruikt zijn op dezelfde manier erop komen. Anders heb je mogelijk verschil in grote van poten, waardoor hij bijvoorbeeld niet meer goed loopt. 

Het hele project lijkt op papier veel minder werk als een normaal project, maar doordat we hem precies moeten namaken heeft dit project meer tijd qua onderzoek gekost als een normaal project. Hiernaast moesten we ook iedere keer controleren of wat we gedaan hadden in dat stukje van Fluffy ook wel klopten met wat er in het orgineel zat. En wanneer dit niet het geval was (Wat zo af en toe wel eens voor kwam), moest dit weer uit elkaar gehaald worden en opnieuw bevestigd. Door dit soort kleine dingen duurt dit project vrij lang.
