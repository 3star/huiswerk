% Presentatie Project 7 (tussentijds)
\documentclass{beamer}

\mode<presentation>

\usepackage[dutch]{babel}
%\usepackage{beamerthemesplit}
\usetheme{Berlin}
\useinnertheme{rounded}
\usecolortheme{rose}

\setbeamertemplate{footline}%[frame number]
{
  \hbox{%
  \begin{beamercolorbox}[wd=.5\paperwidth,ht=2.25ex,dp=1ex,center]{title in head/foot}%
    \usebeamerfont{title in head/foot}\insertshorttitle
  \end{beamercolorbox}%
  \begin{beamercolorbox}[wd=.5\paperwidth,ht=2.25ex,dp=1ex,right]{date in head/foot}%
    \usebeamerfont{date in head/foot}
    \insertframenumber{} / \inserttotalframenumber\hspace*{2ex} 
  \end{beamercolorbox}}%
  \vskip0pt%
}

\setbeamertemplate{navigation symbols}{}

\title{Reprap}
\author{Sebastiaan Polderman (0820738) \and Paul Sohier (0806122)}
\date{\today}

\usepackage{handoutWithNotes}
\pgfpagesuselayout{4 on 1 with notes}[a4paper,border shrink=5mm]

\begin{document}

\frame{
  \titlepage
}

\frame{
  \frametitle{Inhoud}
  \tableofcontents
}

\section{Waarom}
\frame{
  \frametitle{Idee}
  \begin{itemize}
    \item (1) ICT-Delta 2010
    \item (2) Open source
    \item (3) Maak proces
  \end{itemize}
}

\frame{
  \frametitle{Leerdoel}
  \begin{itemize}
    \item (1) Samenwerken (Open source)
    \item (2) Software
    \item (3) Hardware
    \item (4) Documenteren (www.buildingmandel.nl)
  \end{itemize}
}

\frame{
  \frametitle{Voortgang}
  \begin{itemize}
    \item (1) Frame gemaakt
    \item (2) Afstellen
    \item (3) Test prints
  \end{itemize}
}


\frame{
  \frametitle{Doorgave}
  \begin{itemize}
    \item (1) Eerste jaars 
    \item (2) Kennis overdracht
    \item (3) Helpende hand
  \end{itemize}
}


\section{Afsluiting}
\frame{
  \frametitle{Afsluiting}
  
  Zijn er nog vragen?
}

\end{document}
