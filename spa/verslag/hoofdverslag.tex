\hoofdstuk{Opdracht analyse}
Het doel van de opdracht bij spa22 is het onderzoeken van de mogelijkheid tot de verbetering van het inzicht waar minder zelfredzame mensen zich bevinden. Wanneer er een calamiteit is weet de brandweer op het moment niet wanneer of er iemand welke bijvoorbeeld in een rolstoel zit woont in dat pand.

De opdracht is dus kortgezegd om ervoor te zorgen dat de brandweer in een simpel overzicht ziet of er iemand zich bevind in een huis welke extra hulp nodig heeft of bijvoorbeeld op een bepaalde manier reageert wanneer er iets gebeurd. Wanneer de brandweer dit van te voren weet kunnen ze alvast inschatten wat ze kunnen verwachten en of ze bijvoorbeeld extra mensen moeten inzetten om ervoor te zorgen dat ze snel kunnen reageren wanneer ze aankomen.

Een tweede doel van de opdracht is het kijken naar oplossingen waarbij mensen welke bijvoorbeeld doof of slechtziend zijn beter op de hoogte gehouden worden van calimeiten welke in de buurt zijn. Op dit moment is het voor iemand die doof is zeer moeilijk om te weten wat er ergens aan de hand is, aangezien de informatie voorziening op dit moment voornamelijk gebaseerd is op informatie via de radio, wat die mensen uiteraard niet kunnen horen.
Een goed voorbeeld is hierbij de brand in Moerdijk waarbij er op de radio een hoop informatie werd gegeven. Op de TV was er ook informatie, maar doordat de overheid niet verplicht stelt dat TV rijnmond ook als rampenzender moet fungeren is dit geheel op eigen initiatief. TV Rijnmond had in beeld wel een soort van beeldkrant welke iets aan informatie weergaf, maar deze informatie was op zo'n manier summier dat er geen informatie daadwerkelijk uit te halen was.
Op dit moment zijn er wel initatieven om via SMS ook informatie aan te bieden aan mensen in een gebied waar de calimeit is, maar de vraag is dus nog of hierbij iedereen een telefoon heeft en ook daadwerkelijk alle benodigde informatie hieruit kan halen. 
Naast informatie via SMS versturen wordt er ook gekeken naar mogelijkheden om bijvoorbeeld op teletekst van TV Rijnmond informatie neer te zetten wat er aan de hand is.

Er zijn al wel een aantal idee\"{e}n om de informatie voorziening te verbeteren naar minder zelfredzame mensen. De vraag is dus hoe deze informatie naar deze mensen gebracht moet worden en hoe zei de informatie zelf verwachten. Hierbij moet tevens gekeken worden naar wat voor hulpmiddelen hiervoor gebruikt kunnen worden, aangezien niet iedereen bijvoorbeeld een smartphone heeft. Een app voor op een smartphone zal in dit geval dus niet genoeg zijn om daadwerkelijk als oplossing te laten werken.
\hoofdstuk{Opdracht}
Wij hebben besloten om ons primair te gaan richten op het eerste deel van de opdracht en dan specifiek hierbij een deel tot het weergeven van de hoeveelheid minder zelfredzame mensen in een bepaald gebied.
Wat wij willen is dat een hulpverlener bij een grote calamiteit de optie heeft om op een kaart een vorm te kunnen tekenen binnen bijvoorbeeld google maps, en dat de applicatie vanuit hier teruggeeft het aantal minder zelfredzame mensen (En hierbij dan ook tevens wat er aan de hand is, bijvoorbeeld of de persoon in een rolstoel zit, of dat hij bepaalde begeleiding nodig heeft bij een calamiteit. Ook kan in zo'n geval handig zijn om te weergeven welke persoon gewaarschuwd moet worden om te helpen). Aan de hand van deze kaart kan er dan een inschatting gemaakt worden hoeveel mensen er ongeveer nodig zijn om een bepaald gebied te kunnen ontruimen. 

Om dit voor elkaar te krijgen willen we in eerste instantie primair gebruik gaan maken van zoiets als google maps waarbij de dienstverlener een figuur kan aangeven op de kaart en hierbij dus als resultaat de lijst krijgt. Door de privacy redenen en de te koppellen data willen we ons primair richten op de werking van de kaart met de uitvoer en niet op de data welke ingevoerd moet worden. De invoer van deze data zal op een (door ons/andere) gestandaardiseerde manier geimporteerd worden. Door de privacy wet en de beperkte toegangkelijkheid van deze data zal het lastig zijn om te testen met echte data van gebruikers. De data zal in een eindsituatie voornamelijk komen uit een centrale database. Hiernaast moeten mensen ook zelf de mogelijkheid hebben om data op te slaan (Welke tijdelijk of vast is) zonder dat in bijvoorbeeld de basisadministratie staat. 

Bij deze oplossing zijn nog een aantal problemen welke we hebben en die opgelost moeten worden:
\begin{itemize}
        \item Door gebruik te maken van software zoals google maps verplichten we de gebruiker om een internet verbinding te hebben om de data op te halen. Doordat de database met gegevens niet op locatie is, moet dit geen probleem zijn aangezien ook die data via een netwerk opgevraagd moet gaan worden.
        \item We zijn afhankelijk van de werking van een product van een ander bedrijf, niets zegt of die functionaliteit welke wij gebruiken daadwerkelijk ook erin blijft of dat ze deze bijvoorbeeld verwijderen of aanpassen. In zo'n geval zal de software niet langer werken, en aangezien de software maar werkt in een beperkt aantal gevallen zal dit waarschijnlijk pas ondekt worden op het moment wanneer dit ook daadwerkelijk ook gebruikt wordt. 
        \item Doordat google maps niet in alle gevallen compleet up to date is kan het voorkomen dat er een wijziging in bijvoorbeeld een straat is of dat er een straatnaam veranderd is waardoor de kaart niet klopt en de teruggegeven data dus ook niet klopt. Dit is niet echt op te lossen zonder geen software te gebruiken van een ander bedrijf. Als je zelf dit soort software gaat maken (en dus de kaarten bijhouden) ben je hier een veel tijd aan kwijt. Hierbij moet je er zelf ook nog goed voor zorgen dat wanneer iets gewijzigd is je dit echt goed moet updaten en zorgen dat het altijd up to date is. Door gebruik te maken van google maps leg je die verandwoordelijkheid bij het andere bedrijf, welke waarschijnlijk sneller de resources ervoor heeft. Bij het gebruik moet er dan rekening gehouden worden met de kans dat de kaart data niet langer up to date is. 
        \item We moeten nog uitzoeken of google maps wel daadwerkelijk de data welke wij nodig hebben (huisadressen binnen een vlak) ook beschikbaar stelt. Mogelijk is dit niet het geval in verband met hun terms of service (ToS).
\end{itemize}
Voor de meeste problemen hierboven zijn wel oplossingen te bedenken (Of zijn er al een aantal oplossingen), maar er moet nog wel goed nagedacht worden over de complete implementatie van het geheel en ook hoe de interface met de externe database zal gaan werken. Vooral voor de interface zullen we moeten kijken hoe de data momenteel is opgeslagen en hoe wij deze het beste kunnen opvragen zonder de wet op de privacy te overtreden.

Een oplossing voor het probleem met de google maps ToS zou zijn als we uit een map breedtegraad/lengtegraad (latitude/longtitude) co\"{o}rdinaten halen en aan een database presenteren om de ingevoerde data van minder validen te verkrijgen. Het is dan ook mogelijk om die gegevens grafisch weer te geven, of de informatie textgewijs af te lezen, dit lost niet het probleem op met de wet op de privacy. In dit geval wordt er ook niet meer met een adres gezocht in de database, maar met co\"{o}rdinaten.Deze co\"{o}rdinaten zijn zeer accuraat, en er is makkelijk mee te rekenen. Dit is vooral belangrijk voor als er een groot gebied geselecteerd wordt, en alle informatie doorgescant moet worden.