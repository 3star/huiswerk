% Presentatie Project 7 (tussentijds)
\documentclass{beamer}

\mode<presentation>

\usepackage[dutch]{babel}
%\usepackage{beamerthemesplit}
\usetheme{Berlin}
\useinnertheme{rounded}
\usecolortheme{rose}
\setbeamertemplate{navigation symbols}{} 

\title{Hogeschool Rotterdam}
\author{Paul Sohier \and\\Tristan Hakkaart}
\date{\today}

\begin{document}

\frame{
  \titlepage
} 

\frame{
  \frametitle{Inhoud}
  \tableofcontents
}

 \section{De opdracht}
 \frame{
   \frametitle{De opdracht}
   \begin{itemize}
    \item<1-> Verbeteren van informatie voorziening
    \item<2-> Momenteel niets voor geregeld
    \item<3-> Problemen met de Privacy
   \end{itemize}
 }

\section{Oplossing}
\frame{
  \frametitle{Oplossing}
  \begin{itemize}
    \item<1-> Software welke gegevens kan interpreteren
    \item<2-> Gegevens uit diverse bronnen (Gemeente, personen zelf)
    \item<3-> Gegevens worden pas opgehaald bij melding
    \item<4-> Gegevens worden niet lokaal opgeslagen, alleen geprint bij melding
  \end{itemize}
}

\frame{
  \frametitle{Input/Output}
  \begin{itemize}
    \item<1-> Input: postcode, huisnummer, omvang incident
    \item<2-> Output: Alleen relevante gegevens, geen gegevens die namen met iets anders kunnen koppelen
    \item<2-> geslacht, fysieke conditie (Indien relevant), mentale conditie (Indien relevant)
  \end{itemize}
}



\section{Afsluiting}
\frame{
  \frametitle{Afsluiting}
  Zijn er nog vragen?
}

\end{document}
