\chapter{Cryptografie}

\section{Opdracht 1}
\emph{Probeer het volgende Ceasargecodeerde bericht te ontcijferen:
LNNZDPLNCDJHQLNRYHUZRQ}

\section{Opdracht 2}
\emph{Geef enkele voorbeelden waaruit blijkt dat berichten in het algemeen niet uniform verdeeld zijn. Welke oplossing is geschikt om deze berichten uniform verdeeld te maken?}

\section{Opdracht 3}
\emph{Een andere manier om geheime boodschappen te versturen is steganografie.
\begin{itemize}
  \item[(a)] Wanener zouden partijen steganografie gebruiken?
  \item[(b)] Wat is het nadeel van steganografie?
\end{itemize}}

\section{Opdracht 4}
\emph{Een bekende manier om geheime boodschappen te ontcijferen is gebruik te maken van letterfrequenties. Deze methode werkt bij systemen waarbij de letters simpelweg vervangen worden door andere tekens, zoals monoafabetische substituering. Methoden die gebruik maken van verwisselingen van letterposities zijn minder kwetsbaar voor deze methoden.}

\begin{itemize}
  \item[(a)] \emph{Welke principes zijn volgens Shannon noodzakelijk voor een betrouwbaar cryptografisch systeem?}
  \item[(b)] \emph{Als de letters uniform verdeeld zijn in een bericht dan hebben alle letters $i$ = 1\ldots26 evenveel kans om op te treden $p(X=a)=p_a=1/26$. Indien wij een ander bericht van gelijke lengte met willekeurig verdeelde letters op het eerste bericht leggen, dan is de kans dat een positie twee letters 'a' op elkaar liggen gelijk aan $P(X=a\cap x = a)=p_a^2=(1/26)^2=0,0385$. Als wij deze methode per taal uitvoeren, blijkt dat deze co\"{i}ncidentie per letter per taal veschilt. Om deze ereigenschap van een taal met een kental te beschrijven wordt zij gedefineerd als de \emph{co\"{i}ncidentie-index}: $i_c=\sum ^{26}_{i=1} p^2_i$:}

\begin{center}
\begin{tabular}{ll}
  taal & i_c \\
  \hline
  Engels & 00661 \\ 
  Frans & 0,0778 \\
  Duits & 0,0762 \\
  Italiaans & 0,0738 \\
  Japans & 0,0819 \\
  Russische & 0,0529 \\
  Random & 0,0385 \\
\end{tabular}
\end{center}

\emph{Bereken de co\"{i}ncidentie-index voor de Nederlandse taal. Maak gebruik van de gegeven letterfrequenties in bijlage C.}
   \item[(c)] \emph{Hoe zou de co\"{i}ncidentie-index $i_c$ gebruikt kunnen worden bij het kraken van een cipher-text?}
\end{itemize}

\section{Opdracht 5}
\emph{Waarom moet de entropie $H(K|C)$ zo groot mogelijk zijn?}

\section{Opdracht 6}
\emph{Een natuurlijke tekst in het Nederlands heeft een relatieve nulde-orde redundantie van 50\%. De Nederlandse tekst wordt op karakterbasis met een Ceasarcode versleuteld.
\begin{itemize}
  \item[(a)] Bereken de kritieke lengte van de cipher-text.
  \item[(b)] Indien de Nederlandse tekst gecomprimeerd wordt met een code-effici\"{e}ntie van 70\%, wat is dan de kritieke lengte van de Ceasar codering?
\end{itemize}}
