\chapter{Cryptografie}

\section{Opdracht 1 I}
\emph{Probeer het volgende Ceasargecodeerde bericht teontcijferen:\\
LNNZDPLNCDJHQLNRYHUZRQ}

Het ondcijveren van dit Ceasargecodeerde bericht kan onderandere op de volgende manier door alle mogelijke uitkomste op te schrijven tot dat je een leesbaar bericht tegen komt

\begin{tabular}{|c|l|}
  \hline
  0 & LNNZDPLNCDJHQLNRYHUZRQ \\\hline
  1 & KMMYCOKMBCIGPKMQXGTYQP \\\hline
  2 & JLLXBNJLABHFOJLPWFSXPO \\\hline
  3 & IKKWAMIKZAGENIKOVERWON \\\hline
\end{tabular}

In dit geval is de meest logiese uitkomst: Ik kwam ik zag en ik overwon

\section{Opdracht 2 I}
\emph{Geef enkele voorbeelden waaruit blijkt dat berichten in het algemeen niet uniform verdeeld zijn. Welke oplossing is geschikt om deze berichten uniform verdeeld te maken?}

Om een bericht uniform te verdelen is er onderandere een mogelijk hijd om het bedricht in te pakken ofwel comprimeren dit heeft als gevolg dat elk teken of teken reeks gemiddelt even vaak voorkomt.

\section{Opdracht 3 I}
\emph{Een andere manier om geheime boodschappen te versturen is steganografie.}
\begin{itemize}
\item[(a)] \emph{Wanener zouden partijen steganografie gebruiken?} \\
  Als er een open medium is waar het bericht over verstuurt kan woorden waardoor de daatwerkelijke bootschap geheim dient te blijven. Dit is veel gebruikt in het engelse verzet tijdens de orlog. (Er zijn destijds een hoop gijten gemolken)
\item[(b)] \emph{Wat is het nadeel van steganografie?}\\
  Het nadeel van steganografie is dat als het code boek uitlekt iedereen weet wat de berichten betekenen en je beperkt ben tot een code boek om te achterhalen wat een code betekent. je kan namelijk niet alle comandows kwijd in een code boek.
\end{itemize}

\section{Opdracht 4 *}
\emph{Een bekende manier om geheime boodschappen te ontcijferen is gebruik te maken van letterfrequenties. Deze methode werkt bij systemen waarbij dfe letters simpelweg vervangen worden door andere tekens, zoals monoafabetische substituering. Methoden die gebruik maken van verwisselingen van letterposities zijn minder kwetsbaar voor deze methoden.}

\begin{itemize}
  \item[(a)] \emph{Welke principes zijn volgens Shannon noodzakelijk voor een betrouwbaar cryptografisch systeem?}\\
  \ldots

  \item[(b)] \emph{Als de letters uniform verdeeld zijn in een bericht dan hebben alle letters $i$ = 1\ldots26 evenveel kans om op te treden $p(X=a)=p_a=1/26$. Indien wij een ander bericht van gelijke lengte met willekeurig verdeelde letters op het eerste bericht leggen, dan is de kans dat een positie twee letters 'a' op elkaar liggen gelijk aan $P(X=a\cap x = a)=p_a^2=(1/26)^2=0,0385$. Als wij deze methode per taal uitvoeren, blijkt dat deze co\"{i}ncidentie per letter per taal veschilt. Om deze ereigenschap van een taal met een kental te beschrijven wordt zij gedefineerd als de \emph{co\"{i}ncidentie-index}: $i_c=\sum ^{26}_{i=1} p^2_i$:}

\begin{center}
\begin{tabular}{ll}
  taal & i_c \\
  \hline
  Engels & 00661 \\ 
  Frans & 0,0778 \\
  Duits & 0,0762 \\
  Italiaans & 0,0738 \\
  Japans & 0,0819 \\
  Russische & 0,0529 \\
  Random & 0,0385 \\
\end{tabular}
\end{center}

\emph{Bereken de co\"{i}ncidentie-index voor de Nederlandse taal. Maak gebruik van de gegeven letterfrequenties in bijlage C.}
   \item[(c)] \emph{Hoe zou de co\"{i}ncidentie-index $i_c$ gebruikt kunnen worden bij het kraken van een cipher-text?}\\

\end{itemize}

\section{Opdracht 5 I}
\emph{Waarom moet de entropie $H(K|C)$ zo groot mogelijk zijn?}\\
De entropie dient zo hoog mogelijk te zijn zodat als C bekent is dat dan de K niet makkelijk gevonden.

\section{Opdracht 6 *}
\emph{Een natuurlijke tekst in het Nederlands heeft een relatieve nulde-orde redundantie van 50\%. De Ne-der-landse tekst wordt op karakterbasis met een Ceasarcode versleuteld.}\\
\begin{itemize}
  \item[(a)] \emph{Bereken de kritieke lengte van de cipher-text.}\\

  \item[(b)] \emph{Indien de Ne-der-landse tekst gecom-primeerd wordt met een code-effici\"{e}ntie van 70\%, wat is dan de kritieke lengte van de Ceasar codering?}\\
\end{itemize}
