\documentclass[a4paper,11pt]{report}
% Hier hebben we de preamble, alle document settings moeten hier:
\usepackage{graphicx}
\usepackage{url}
\usepackage{appendix}
\usepackage[titles]{tocloft}
\usepackage[dutch]{babel}
\usepackage{listings}
\usepackage{makeidx}
\usepackage{float}
\usepackage[hypertexnames=false]{hyperref}

% Custom LaTeX commands:
% ~ == {\raise.17ex\hbox{$\scriptstyle\sim$}}
\newcommand{\customtilde}{\raise.17ex\hbox{$\scriptstyle\sim$}}
% Meta info pdf:
\hypersetup{
%bookmarks=true, % show bookmarks bar?
unicode=false, % non-Latin characters in Acrobat’s bookmarks
pdftoolbar=true, % show Acrobat’s toolbar?
pdfmenubar=false, % show Acrobat’s menu?
pdffitwindow=false, % window fit to page when opened
%pdfstartview={FitH}, % fits the width of the page to the window
pdftitle={Project 7/8}, % title
pdfauthor={Paul Sohier Sebastiaan Polderman}, % author
pdfsubject={Project 7/8}, % subject of the document
pdfcreator={make}, % creator of the document
pdfproducer={make}, % producer of the document
pdfkeywords={Linux} {Basis}, % list of keywords
pdfnewwindow=true, % links in new window
colorlinks=false, % false: boxed links; true: colored links
linkcolor=black, % color of internal links
citecolor=green, % color of links to bibliography
filecolor=magenta, % color of file links
urlcolor=cyan % color of external links
}

% Paragrafen hebben een witregel ertussen, en geen indent tab:
\setlength{\parindent}{0.0in}
\setlength{\parskip}{0.1in}
% Onderstaande is voor de dots tussen chapter title + blz. 
\makeatletter
\renewcommand*\l@chapter[2]{%
  \ifnum \c@tocdepth >\m@ne
    \addpenalty{-\@highpenalty}%
    \vskip 1.0em \@plus\p@
    \setlength\@tempdima{1.5em}%
    \begingroup
      \parindent \z@ \rightskip \@pnumwidth
      \parfillskip -\@pnumwidth
      \leavevmode \bfseries
      \advance\leftskip\@tempdima
      \hskip -\leftskip
      #1\nobreak\normalfont\leaders\hbox{$\m@th
        \mkern \@dotsep mu\hbox{.}\mkern \@dotsep
        mu$}\hfill\nobreak\hb@xt@\@pnumwidth{\hss #2}\par
      \penalty\@highpenalty
    \endgroup
  \fi}
\makeatother

% End of title + blz.
% Table of content depth van 4, dus tm paragraph
\setcounter{tocdepth}{4}
%\renewcommand{\baselinestretch}{1.5} 1.5 regelafstand 

%Pas listings aan zodat ze duidelijker zijn
\lstset{ %
  language=bash,                % choose the language of the code
  basicstyle=\footnotesize,       % the size of the fonts that are used for the code
  numbers=left,                   % where to put the line-numbers
  numberstyle=\footnotesize,      % the size of the fonts that are used for the line-numbers
  numbersep=5pt,                  % how far the line-numbers are from the code
  showspaces=false,               % show spaces adding particular underscores
  showstringspaces=false,         % underline spaces within strings
  showtabs=false,                 % show tabs within strings adding particular underscores
  frame=lr,	                % adds left and right lines
  tabsize=2,	                % sets default tabsize to 2 spaces
  captionpos=b,                   % sets the caption-position to bottom
  breaklines=true,                % sets automatic line breaking
  breakatwhitespace=false,        % sets if automatic breaks should only happen at whitespace
%  escapeinside={\%*}{*)},         % if you want to add a comment within your code
  morekeywords={*,...}            % if you want to add more keywords to the set
}
%hyperref aanpassingen
\hypersetup{pdfborder=0 0 0}
% We gebruiken de index
\makeindex

% Einde preamble, begin document; 
\begin{document}
% Title page
\title{
  Project 7/8 eindverslag
}
\author{
  Sebastiaan Polderman\\
  0820738
  \and
  Paul Sohier\\
  0806122
}
\date{\today}
% Print de title
\maketitle

% Abstract (+ in toc)?:
%\begin{abstract}\centering

%\end{abstract}

% De table of contents + toc in toc:
\tableofcontents
\addcontentsline{toc}{chapter}{\numberline{}Inhoudsopgave}

% HACK: Page number fixen. Dit is voor makeidx + hyperref
% 15 is de pagina met hoofdstuk 01, dus inleiding erop. 
% Dit zorgt ervoor dat alle
%\setcounter{page}{3}
% Nu kunnen we de losse hoofdstukken gaan includen. 
% Includen gebeurt met basename, dus zonder .tex
\chapter{Inleiding}
Dit document gaat over het eindresultaat van Project 7 en 8. Wij hebben Project 7 en 8 samengevoegd tot \'{e}\'{e}n groot project waarbij wij \'{e}\'{e}n einddoel hadden. Dit project is hierbij weer verdeeld in diverse kleinere onderdelen. 

In dit document gaan we kort uitleggen wat ons project precies inhield en welke we problemen we hierbij tegenkwamen. Door deze problemen zijn wij helaas niet tot een werkend eindproduct gekomen, maar dit zullen wij verderop in dit document uitgebreid toelichten.

\chapter{De bouw}
Doordat we geen compleet eigen ontwerp gingen maken van Fluffy hebben we eerst helemaal uitgezocht hoe Fluffy precies in elkaar zit. Zonder deze informatie kunnen we hem niet namaken, en doordat we de code niet hebben moeten we er ook voor zorgen dat hij ook goed in elkaar zit zoals de orginele Fluffy. 

Het belangrijkste hiervan zijn de motoren. De motoren worden aangestuurd via het nummer van de motor. Iedere motor heeft een uniek nummer in de robot. Via dit nummer wordt die motor aangestuurd. Wanneer de motor opeens op een andere plek zou zitten als in de orginele Fluffy gaat hij mischien wel lopen in plaats van met zijn staart te kwispellen. En dit willen we uiteraard niet zien gebeuren. 
Naast de motoren moesten we er ook voor zorgen dat alle plastic onderdelen die verder gebruikt zijn op dezelfde manier erop komen. Anders heb je mogelijk verschil in grote van poten, waardoor hij bijvoorbeeld niet meer goed loopt. 

Toen we een complete inventaris hadden van hoe fluffy compleet in elkaar zet konden we beginnen met het bouwen van Fluffy. Het bouwen zelf heeft eigenlijk minder tijd gekost als het complete onderzoek naar Fluffy. We liepen bij de bouw wel met regelmaat tegen allerlei problemen op. Dit bestond voornamelijk uit dat de motoren of plastic onderdelen verkeerd om zaten of het verkeerde onderdeel gebruikt was.

Het hele project lijkt op papier veel minder werk als een normaal project, maar doordat we hem precies moeten namaken heeft dit project meer tijd qua onderzoek gekost als een normaal project. Hiernaast moesten we ook iedere keer controleren of wat we gedaan hadden in dat stukje van Fluffy ook wel klopten met wat er in het orgineel zat. En wanneer dit niet het geval was (Wat zo af en toe wel eens voor kwam), moest dit weer uit elkaar gehaald worden en opnieuw bevestigd. Door dit soort kleine dingen duurt dit project vrij lang.

Omdat Fluffy een niet hele grote hond is hebben we eigenlijk in bijna alle gevallen alleen gewerkt. Dit om ervoor te zorgen dat we elkaar niet in de weg zaten en hierdoor dus nog meer fouten gingen maken bij de bouw. Wel konden we bijvoorbeeld tegelijk werken, maar dan allebei aan een ander onderdeel van Fluffy. Pas op het eind toen we alle lossen onderdelen (Poten, staart, lijf) bij elkaar gingen voegen hebben we met zijn twee\"{e}n aan heel Fluffy gewerkt.

\chapter{Hoofdstuk 3}

\section{Opdracht 1}
Een BSC heeft een \emph{Binary Error Rate} van 0,3. Wat is de discrete kanaalcapaciteit van dit kanaal?

\section{Opdracht 2}
Noem 3 decodeerprincipes en hun eigenschappen.

\section{Opdracht 3}
De \emph{Soundex codering} is een codering die bronwoorden uit een West Europesche spreektaal vertaalt in codewoorden van \'{e}\'{e}n letter gevolgd door drie cijfers. Het voordeel van deze codering is dat de woorden is dat de woorden die veel op elkaar lijken qua uitspraak, dezelfde code krijgen. Soundex wordt veel toegepast in spellingscontrole in woordproccesssoren, reisplanners en reserveringssystemen etc. Bijvoorbeeld, de namen \emph{'brok', 'brock', 'broek'} geven dezelfde code B620. Daarentegen geven de namen \emph{'jansen', 'janssen', 'jansens'} de \emph{J525}.

Het soundex-algoritme werkt als volgt:
\begin{itemize}
  \item[(a)]De eerste letter van het bronwoord wordt de beginletter in het Soundex codewoord;
  \item[(b)]De volgende 3 cijfers komen uit de volgende tabel. Zijn worden in volgorde opgebouwd in volgorde van de letters in het bronwoord. Als whet bronwoord te kort is voor een volledig Soundex codewoord, wordt het Soundex codewoord aangevuld met nullen. Te lange codewoorden worden afgebroken na drie cijfers.
\end{itemize}

\begin{tabular}{lll}
code & letter & uitspraakorgaan \\
\hline
1 & b p f v & lippen\\
2 & c s k g j q x z & keel\\
3 & d t & tanden\\
4 & l & tong voor\\
5 & m n & neus \\
6 & r & tong achter
geen & a e h i o u y w & \\
\end{tabular}
Deze tabel geeft de cijfercodering van de letters aan. De meeste medeklinkers zijn volgens uitspraak gegroepeerd. De klinkers en de zachte medeklinkers krijgen geen cijfercode. Hoewel het Soundex-algoritme goed werkt voor West-Europese talen, is het niet voor andere spreektalen geschikt.

\begin{itemize}
  \item[(a)] Welke aspecten maken Soundex codering geschikt voor Weste-Europese talen?
  \item[(b)] Een andere manier om alternatieve woorden te vinden is de '\emph{Edit Distance}':
    \begin{itemize}
      \item Spatie tussenvoegen;
      \item Twee buurletters verwisselen;
      \item Een letter vervangen door een andere;
      \item Het verwijderen van een letter;
      \item Een letter toevoegen.
    \end{itemize}
    Wat is het belanrijkste verschil tussen de 'Edit Distance' en Soundex?
\end{itemize}

\section{Opdracht 4}
Gegeven een taal met 8 bronwoorden van 3 bit.
\begin{itemize}
  \item[(A))] Hoeveel redundantie moeten de codewoorden bevatten om een taal met 8 bronwoorden van ieder 3 bit intolerant voor 2 bit fouten te maken?
\item[(b)] Welke CRC-polynoom zou in aanmerking komen om de bronwoorden te beschermen tegen 1 en 2 bit fouten?
\end{itemize}

\section{Opdracht 5}
Een geheugenloos kanaal tussen \emph{X} en \emph{Y} heeft de volgende eigenschappen $P(Y=y|X=x)=0,5$
\begin{itemize}
  \item[(a)] Bepaal de discrete kanaalcapaciteit van het kanaal in figuur 5.5?
  \item[(b)] Ontwerp een transmissiecode waarmee via dit kanaal foutloze transmisse plaats kan vinden.
\end{itemize}

\chapter{Cryptografie}

\section{Opdracht 1}
Probeer het volgende Ceasargecodeerde bericht teontcijferen:
LNNZDPLNCDJHQLNRYHUZRQ

\section{Opdracht 2}
Geef enkele voorbeelden waaruit blijkt dat berichten in het algemeen niet uniform verdeeld zijn. Welke oplossing is geschikt om deze berichten uniform verdeeld te maken?

\section{Opdracht 3}
Een andere manier om geheime boodschappen te versturen is steganografie.
\begin{itemize}
  \item[(a)] Wanener zouden partijen steganografie gebruiken?
  \item[(b)] Wat is het nadeel van steganografie?
\end{itemize}

\section{Opdracht 4}
Een bekende manier om geheime boodschappen te ontcijferen is gebruik te maken van letterfrequenties. Deze methode werkt bij systemen waarbij dfe letters simpelweg vervangen worden door andere tekens, zoals monoafabetische substituering. Methoden die gebruik maken van verwisselingen van letterposities zijn minder kwetsbaar voor deze methoden.

\begin{itemize}
  \item[(a)] Welke principes zijn volgens Shannon noodzakelijk voor een betrouwbaar cryptografisch systeem?
  \item[(b)] Als de letters uniform verdeeld zijn in een bericht dan hebben alle letters $i$ = 1\ldots26 evenveel kans om op te treden $p(X=a)=p_a=1/26$. Indien wij een ander bericht van gelijke lengte met willekeurig verdeelde letters op het eerste bericht leggen, dan is de kans dat een positie twee letters 'a' op elkaar liggen gelijk aan $P(X=a\cap x = a)=p_a^2=(1/26)^2=0,0385$. Als wij deze methode per taal uitvoeren, blijkt dat deze co\"{i}ncidentie per letter per taal veschilt. Om deze ereigenschap van een taal met een kental te beschrijven wordt zij gedefineerd als de \emph{co\"{i}ncidentie-index}: $i_c=\sum ^{26}_{i=1} p^2_i$:

\begin{center}
\begin{tabular}{ll}
  taal & i_c \\
  \hline
  Engels & 00661 \\ 
  Frans & 0,0778 \\
  Duits & 0,0762 \\
  Italiaans & 0,0738 \\
  Japans & 0,0819 \\
  Russische & 0,0529 \\
  Random & 0,0385 \\
\end{tabular}
\end{center}

Bereken de co\"{i}ncidentie-index voor de Nederlandse taal. Maak gebruik van de gegeven letterfrequenties in bijlage C.
   \item[(c)] Hoe zou de co\"{i}ncidentie-index $i_c$ gebruikt kunnen worden bij het kraken van een cipher-text?
\end{itemize}

\section{Opdracht 5}
Waarom moet de entropie $H(K|C)$ zo groot mogelijk zijn?

\section{Opdracht 6}
Een natuurlijke tekst in het Nederlands heeft een relatieve nulde-orde redundantie van 50\%. De Nederlandse tekst wordt op karakterbasis met een Ceasarcode versleuteld.
\begin{itemize}
  \item[(a)] Bereken de kritieke lengte van de cipher-text.
  \item[(b)] Indien de Nederlandse tekst gecomprimeerd wordt met een code-effici\"{e}ntie van 70\%, wat is dan de kritieke lengte van de Ceasar codering?
\end{itemize}


% Einde document
\end{document}
