\hoofdstuk{Voorwoord}

Dit verslag is geschreven naar aanleiding van onderzoek naar integratie mogelijkheden van Maple TA op de HR. Tevens dient dit verslag een inzicht te bieden op de uitgevoerde werkzaamheden aan de beoordelende docent en de `opdrachtgever'.

Met het uitgevoerde onderzoek wordt de invulling gegeven aan minor project TIRBPF02A. In het vervolgproject TIRBPF02B wordt de gewonnen informatie en geschreven handleidingen toegepast om de het geheel te implementeren. Verder in dit verslag zal de volledige opdracht omschrijving volgen. 

Om dit rapport op te kunnen leveren is er veel onderzoek uitgevoerd naar de benodigdheden, maar ook de mogelijkheden van zo wel systemen van de HR als van de Maple TA. 

Bij het onderzoeken van de mogelijkheden en het opzetten van test opstellingen heb ik inhoudelijke hulp gehad van mijn medestudent, Kevin van der Vliest. Verder heb ik dankzij ICT medewerker van de HR, Jeffrey Sleddens, toegang gekregen tot de systemen die van cruciaal belang voor het testen waren. Beide heren wil ik hartelijk bedanken voor alle hulp, tijd en moeite die ze erin gestoken hebben. Verder wil ik graag mijn begeleidende docent, meneer den Brok, bedanken voor het steeds achter de broek aan zitten. Ondanks alle drukte heeft hij me geholpen het zo ver te brengen en dit onderzoek met een verslag af te ronden.

Zo als ik al zei was ik niet de hele tijd met de opdracht bezig. Dit in verband met andere, even belangrijke werkzaamheden voor de HR. Verder was het af en toe moeilijk om sommige mensen te bereiken i.v.m. de vakanties en verschillende werkroosters. Maar over het algemeen genomen ging het goed. Ik heb heel veel geleerd tijdens het opzetten van de testopstellingen. Ook heb ik veel handige connecties gelegd.

In het algemeen kan ik zeggen dat ik het heel erg leuk vond om dit onderzoek uit te voeren.