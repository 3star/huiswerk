\hoofdstuk{Inleiding}

Op dit moment gebruikt de \HR{} diverse digitale systemen zoals
N@tschool om lesmateriaal te verstrekken en projectactiviteiten van
studenten te ondersteunen, Osiris om de studieresulaten te
registreren, Confluent voor de wiki-activiteiten en MapleTA om
wiskundige- en technische vakken te toetsen.

Voor bijna alle bovengenoemde systemen is het inloggen
noodzakelijk. Daarom is er een centraal loginpunt gemaakt. Die is te
vinden op \url{http://login.hro.nl}. De digitale systemen hebben een
plug-in die kijkt of de gebruiker ingelogd is op het centrale login
punt; als dit wel het geval is, wordt men doorgestuurd naar de
hoofdpagina van het opgevraagde systeem. Is er nog niet ingelogd, dan
wordt men eerst naar \url{http://login.hro.nl} doorgestuurd. Na het
aanmelden wordt men vanzelfsprekend naar de hoofdpagina van het
opgevraagde systeem doorgestuurd.

Het grootste voordeel van dit systeem is dat de gebruiker maar \'e\'en
maal op het centrale loginpunt hoeft in te loggen. Daarna heeft hij
de toegang tot alle systemen die er aan gekoppeld zijn.

MapleTA maakt echter nog geen gebruik van het centrale loginpunt. Dit
brengt een aantal ongemakken met zich mee. De grootste is de navigatie
binnen MapleTA. Die is zeer ingewikkeld voor personen die nog onbekend
zijn met het systeem. Studenten kunnen niet de toets vinden die bij
een vak hoort en docenten hebben zo veel vakken dat ze per ongeluk de
verkeerde toetsen kunnen wijzigen.

Een denkbare oplossing hier voor is het koppelen van toetsen van een
bepaald vak aan het overeenkomstige vak in N@tschool. Het enige
probleem daarbij is het feit dat MapleTA niet aan het centrale login
punt gekoppeld is. Daardoor werken de links vanuit N@tschool niet.

Tijdens dit onderzoek werd er gekeken naar de mogelijkheid om MapleTA
door middel van `SingleSignOn' aan het centrale loginpunt te
koppelen. Verder werd er gekeken of het niet handiger was om van
N@tschool naar BlackBoard over te stappen, gezien het feit dat voor
BlackBoard al een koppelingsmodule voor MapleTA bestaat. Ook moest er
een eisenpakket opgesteld worden voor de navigatievoorziening. Het
onderzoek moet de basis zijn voor het architectonisch ontwerp van de
navigatievoorzieningen.

Ook is de mogelijkheid onderzocht om de cijferregistratie van MapleTA
te koppelen aan Osiris. Ook daarvoor moest er een eisenpakket en het
architectonische ontwerp gemaakt worden. 

In de hoofdstukken hierna wordt als eerste de gebruikte
onderzoeksmethode beschreven. Daarna worden de problemen die tijdens
de opdracht naar voren kwamen behandeld. Soms waren er meer
oplossingen voor één probleem mogelijk, dus er moesten keuzes gemaakt
worden. Ook die keuzes worden toegelicht in een apart
hoofdstuk. Uiteindelijk komen de conclusies en aanbevelingen aan de
orde.
