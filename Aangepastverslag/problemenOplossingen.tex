\hoofdstuk{Problemen en oplossingen}

Tijdens dit onderzoek ben ik regelmatig tegen problemen
aangelopen. Sommige van die problemen kon ik niet zelf oplossen. In
die gevallen moest ik hulp vragen bij andere mensen. Sommige van de
problemen konden op meerdere manieren opgelost worden.

\paragraaf{Het SSL-certificaat}

Voor het onderzoek heb ik een aparte Linuxserver ingericht. Daar kon
ik nieuwe scripts uitproberen en verschillende oplossingen testen. Een
van de basis problemen waar ik tegenaan botste was mijn geringe kennis
van Linux en de shell. Gelukkig baart oefening kunst en heb ik heel
wat geleerd. Een ander probleem was het configureren van LDAP voor
MapleTA. Eerder heeft de verantwoordelijke medewerker van de
ICT-afdeling mijn testserver toegang tot LDAP gegeven. Ik dacht dat
het daarmee klaar was, en de rest in de LDAP-configuratiefile
ingesteld moest worden. Helaas was dit niet het geval. Om gebruik te
maken van LDAP-server van school moest er ook in de testomgeving
gebruik gemaakt worden van de beveiligde verbinding. Daarvoor moest ik
het SSL-certificaat, de keystore en de truststore file van de \HR{}
hebben.

Gezien het feit dat het om de beveiliging van de \HR{} files ging moest
ik eerst op toestemming van de desbetreffende afdeling
wachten. Uiteindelijk mocht ik de certificaten gebruiken, maar ik
mocht het niet zelf installeren. Het werd voor me gedaan.

\paragraaf{De \HR{} en het MBO}
\label{sec:MBO} 

De volgende problemen waren geen echte problemen, maar vraagstukken en
deelvragen die aan de hand van de verschillende gesprekken beantwoord
moesten worden:

Vanuit de opleiding was er een vraag over wenselijkheid van de
koppeling van de MapleTA server via LDAP aan het centrale
loginpunt. Enige nadeel daarvan is dat de MBO-studenten er niet meer
bij kunnen, omdat ze nog niet bij de \HR{} ingeschreven zijn. Daarvoor
zijn er de volgende oplossingen mogelijk.

\begin{description}
\item[Hoeft niet in te loggen:] studenten hoeven niet in te
  loggen. Voordeel hiervan is dat de MBO studenten erbij kunnen. Een
  grote nadeel is dat de tentamens ook altijd open zijn en studenten
  dus die van te voren kunnen maken. Een andere nadeel is dat bij de
  anonieme inlog, er geen resultaten worden bijgehouden. Daarmee
  verdwijnt een heel belangrijk en noodzakelijk eigenschap van
  MapleTA;
\item[Moet wel inloggen, maar niet via LDAP:] dat wil zeggen dat, in
  plaats van de centrale database met de ingeschreven studenten,
  MapleTA een eigen database met de gebruikers bijhoudt. De
  administrator kan dan alle gebruikers handmatig toevoegen, de
  student hoeft daarvoor niet bij de \HR{} te studeren. Nadeel van deze
  methode is de navigatie. Als er geen gebruik wordt gemaakt van LDAP,
  kan MapleTA dus ook niet aan het centrale loginpunt gekoppeld
  worden, en blijft de navigatie onduidelijk;
\item[Twee servers:] er kunnen twee servers opgezet worden. Eentje
  voor de tentamens, die zal via LDAP werken en alleen voor de
  studenten van de \HR{} bereikbaar zal zijn. En een server met een
  eigen gebruikersdatabase. Zo wordt de tentamenserver aan N@tschool
  gekoppeld en hoeven de studenten niet tijdens de toets naar het
  juiste vak te zoeken en daarmee hun tijd te verspillen. Server twee
  kan dan gebruikt worden om te oefenen, zo wel voor eigen studenten
  als voor de MBO-studenten. Server twee moet overigens wel een eigen
  database hebben, anders worden de resultaten van de oefentoetsen
  niet bijhouden.
\end{description}

\paragraaf{Navigatie}

Een deelvraag die hieruit volgt is de navigatie binnen de
MapleTA. Gezien de ervaringen van docenten en studenten ermee, bestaat
er een noodzaak om het geheel overzichtelijker te maken. Het zou op
verschillende manieren kunnen.

\begin{description}
\item[Overstappen op BlackBoard.] De makers van MapleTA hebben een
  plug-in geschreven die zorgt voor de koppeling tussen beide
  systemen. Voordeel: het geheel wordt overzichtelijk, zonder dat er
  veel tijd in de ontwikkeling van een nieuwe plug-in voor N@tschool
  gestoken moet worden. Nadeel: Het is weer een nieuwe leeromgeving,
  waarmee docenten en studenten moeten leren werken. Bovendien moeten
  er dan licenties voor BlackBoard gekocht worden, terwijl de
  licenties voor N@tschool al betaald zijn;
\item[Eigen interface schrijven.] Één van de mogelijkheden is het zelf
  schrijven van een interface voor MapleTA. Voordeel: deze kan zo
  overzichtelijk gemaakt worden als men wilt. Nadeel: het zal een heel
  groot project worden en te zwaar voor een
  afstudeerproject. Bovendien zou men meerdere programmeurs nodig
  hebben;
\item[Koppelen aan de centrale inlog punt.] Als MapleTA `SSO'
  ondersteunt, zou de navigatie rechtstreeks vanuit N@tschool kunnen
  plaatsvinden. Verder zouden docenten ook direct vanuit hun website
  naar de oefentoetsen kunnen verwijzen. Nadeel: niet iedereen op de
  \HR{} is overtuigd van de nuttigheid van N@tschool. Bovendien zijn
  er nog een aantal docenten die er niet mee bekend zijn.
\end{description}

\paragraaf{Cijferregistratie}

De volgende deelvraag is de mogelijkheid om cijfers vanuit MapleTA te
transporteren naar Osiris. Na de afloop van een kwartaal krijgen
docenten een mail van het Osiristeam. In die mail zit een Excelbestand
volgens een bepaald formaat met alle namen van studenten die dit vak
hadden moeten volgen. Het is de bedoeling dat de docenten dit
excelbestand vullen en terug sturen naar het Osiris team. Het geheel
gebeurd volgens een beveiligde procedure. Meeste docenten brengen de
cijfers handmatig over. Sommigen gebruiken de exportfunctie van
MapleTA en krijgen een csv-file. Die file parsen ze zelfstandig naar
een Excelformaat en kopiëren resultaten in de Excelsheet van het
Osiris team. Om het geheel te automatiseren en toch de huidige
beveiliging te behouden zijn er twee mogelijkheden.

\begin{description}
\item[Aparte server] voor de data opzetten. In MapleTA komt dan een
  extra knop waarmee docenten cijferlijst kunnen uploaden op de
  server. Daar wordt de data in het juiste formaat omgezet en kunnen
  de medewerkers van het Osiris team het gewoon downloaden. Tijdens
  het uploaden krijgt de docent een aantal controle stappen net zoals
  het momenteel het geval is;
\item[Script] voor het omzetten van cijfers naar het juiste formaat
  direct op de MapleTA server draaien. En in MapleTA gewoon een extra
  knop maken die de cijfers inleest, naar het juiste formaat omzet, en
  daarna een uitvoer geeft in de vorm van een excelbestand. In
  hetzelfde formaat als het Osiris team het wilt.
\end{description}

\paragraaf{Proxy instellingen}
\label{sec:proxy} 

Tijdens het testen ontdekte ik dat Tomcat tijdens de installatie van
MapleTA verkeerd was geconfigureerd. Om dat op te lossen werd de hele
server opnieuw geïnstalleerd. Daarna moest er een proxyserver
opgezet worden die het internet verkeer die naar de server van MapleTA
zoekt moest omleiden naar een specifiek webadres.

Het is gelukt om de proxyserver op de gebruikelijke manier op te
zetten, maar het is nog steeds niet gelukt om het werkend te
krijgen. Op de een of andere manier werkt exact hetzelfde stukje code
wel op de ene server en niet op de andere server.

Wel denk ik te weten waar het probleem aan ligt. MapleTA probeert zelf
gebruikers door te verwijzen naar een ander adres en op deze plek
kunnen die twee instellingen conflicteren met elkaar. Tegen de tijd
dat dit verslag geschreven moest worden, was het probleem nog niet
opgelost. Maar het is geen bedreiging voor het project, omdat in de
logfiles van de proxy en in de logfiles van MapleTA opgezocht kan
worden wat er fout gaat en gerepareerd moet worden. Echter had ik daar
nog geen tijd voor, want daarvoor moet je nog dieper in de Apache
documentatie \cite{apache2} duiken.

Dat waren de belangrijkste problemen en vraagstukken waar ik tegenaan
ben gelopen. In het volgende hoofdstuk worden de uiteindelijke keuzes
besproken en onderbouwd.
