\hoofdstuk{Conclusies en aanbevelingen}

Het is zeker mogelijk om MapleTA aan de huidige versie van de
leeromgeving N@tschool te koppelen. Dat kan aangetoond worden aan de
hand van de verschillende testen op de testserver. Omdat N@tschool wel
gekoppeld kan worden aan MapleTA is het niet verstandig naar een
andere leeromgeving zoals BlackBoard over te stappen. Het zou teveel
geld en vooral teveel tijd kosten.

Om MBO studenten die niet kunnen inloggen, `bij te spijkeren' met
oefentoetsen, is het mogelijk een tweede MapleTA server op te
zetten. Indien de resultaten van de oefentoetsen belangrijk zijn om te
bewaren, moet het een server met een eigen database zijn. Anders kan
het inlogsysteem volledig worden uitgeschakeld zodat iedereen bij
de oefentoetsen kan komen.

Het is ook mogelijk om cijferregistratie te vereenvoudigen voor
docenten en in ieder geval het zelf omzetten naar het juiste datatype
ze uit de handen te nemen. Helaas kan het opsturen naar het Osiris
team niet helemaal automatisch gebeuren, omdat er ook aan de
veiligheidsregels (WHW en HR) voldaan moet worden. Zo moet het Osiris
team het zeker weten dat de desbetreffende cijferlijst van de juiste
docent afkomstig is. Ook moet de docent het ondertekenen met zijn
persoonlijke digitale hndtekening. En als extra controle moet het
`volgnummer' in de mail overeenkomen met het `volgnummer' op het
papieren exemplaar.

Tijdens het onderzoek is er ook naar voren gekomen dat de HR nog
steeds versie 4 van MapleTA gebruikt, terwijl versie 5 al lang uit
is. Intussen is zelfs versie 6 uit, en versie 7 wordt in maart/april
2011 verwacht. Het is zeer aan te raden om de software te blijven
updaten, omdat MapleTA wordt gebruikt door heel veel universiteiten,
hogescholen en middelbare scholen. Elke gebruiker die iets mist in de
lay-out, navigatie of in de mogelijkheden van MapleTA kan dat
doorgeven aan de ontwikkelaars. En er wordt zeker naar geluisterd. Zo
is bijvoorbeeld in versie 6 het `gradebook' sterk verbeterd. Er zijn
een aantal nieuwe mogelijkheden bijgekomen.

De belangrijkste conclusie van dit haalbaarheidsonderzoek luidt:

\begin{quote} Het is zeker mogelijk en haalbaar om MapleTA in de
  huidige digitale leeromgeving van de \HR{} te integreren.
\end{quote}

Daarom beveel ik u aan om de projecten TIRBPF2B en CMIAFST1 door te
laten gaan op basis van de resultaten van dit onderzoek.
