\hoofdstuk{Evaluatie}

Deze onderzoeksopdracht betrof de mogelijkheden en onmogelijkheden van
systemen zoals MapleTA binnen de beperkingen van een
onderwijsorganisatie zoals de \HR{}.

Ik was niet de hele tijd met de opdracht bezig in verband met andere,
even belangrijke werkzaamheden voor de \HR{}. Verder was het af en toe
moeilijk om sommige mensen te bereiken i.v.m. de vakanties en
verschillende werkroosters. Maar over het algemeen genomen ging het
goed. Ik heb heel veel geleerd tijdens het opzetten van de
testopstellingen. Ook heb ik veel handige connecties gelegd.

In het algemeen kan ik zeggen dat ik het heel erg leuk vond om dit
onderzoeksproject uit te voeren. Daarom heb ik ook zo veel geleerd uit
dit project. Ik heb nu veel meer kennis van Linux en van werken met
servers. Daarnaast heb ik heel veel nuttige contacten met mensen uit
de binnen en buiten de organisatie kunnen leggen. Daarbij heb ik
geleerd hoe ik moet onderhandelen en organiseren. Ook heb ik moeten
experimenteren om de haalbaarheid van bepaalde ideeën te toetsen.  Het
onderzoeksrapport opstellen was een nieuwe ervaring voor me. Als ik
de Dublin descriptoren \cite{descriptors} beschouw, dan vind ik dat
aan de volgende kwalificaties gewerkt heb:

\begin{description}
\item[Kennis en inzicht:] Heeft aantoonbare kennis en inzicht van een
  vakgebied, waarbij wordt voortgebouwd op het niveau bereikt in het
  voortgezet onderwijs en dit wordt overtroffen; functioneert
  doorgaans op een niveau waarop met ondersteuning van
  gespecialiseerde handboeken, enige aspecten voorkomen waarvoor
  kennis van de laatste ontwikkelingen in het vakgebied vereist is;
\item[Oordeelsvorming:] Is in staat om relevante gegevens te
  verzamelen en interpreteren (meestal op het vakgebied) met het doel
  een oordeel te vormen dat mede gebaseerd is op het afwegen van
  relevante sociaal-maatschappelijke wetenschappelijke of ethische
  aspecten;
\item[Communicatie:] Is in staat om informatie, ideeën en oplossingen
  over te brengen op een publiek bestaande uit specialisten of
  niet-specialisten.
\item[Leervaardigheden:] Bezit de leervaardigheden die noodzakelijk
  zijn om een vervolgstudie die een hoog niveau van autonomie
  veronderstelt aan te gaan.
\end{description}

Volgende kwalificatie is niet volledig behaald, omdat het project nog
niet geïmplementeerd is. Natuurlijk zijn er al wat experimenten
geweest, maar dat vind ik nog onvoldoende om hieraan te voldoen.

\begin{description}
	\item[Toepassen kennis en inzicht:] Is in staat om zijn/haar
kennis en inzicht op dusdanige wijze toe te passen, dat dit een
professionele benadering van zijn/haar werk of beroep laat zien, en
beschikt verder over competenties voor het opstellen en verdiepen van
argumentaties en voor het oplossen van problemen op het vakgebied.
\end{description}
