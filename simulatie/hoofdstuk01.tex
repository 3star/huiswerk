\chapter{Casus 3}
\section{Opdracht 1}
Opstapeling wordt veroorzaakt doordat er reeds gemaakte producten moeten
wachten op een product dat van de zijkant wordt opgezet. Opstapeling zou
dus maximaal moeten vormen bij het punt met de hoogste opzettijd (de
opzetlocatie van machine 4). 

Met X zijnde de productietijd: 

In de 10 seconde opzettijd dat er gewacht moet worden, worden er
(10/X)*3 producten opgestapeld. Na het opzetten duurt het X-10 seconden
voor er een nieuw product wordt opgezet, dus zijn er 10-X seconden om de
producten door te sturen.

(10/X)*3+1 producten moeten vervoert worden in X-10 seconden. 

30/X+1<X-10

30/X+11<X

30/X+11-X<0

-30/X-11+X>0

X^{2}-11X-30>0

X>13.26

(3600/13.26) = 271 dozen per uur per machine 

Wordt de productie groter dan dit, zal het systeem vastlopen bij machine
3, omdat er op de hoofdband een opstapeling gevormd wordt die het
systeem niet kan verwerken voordat er een nieuwe doos in de wachtrij
wordt gezet om opgezet te worden. 
\section{Opdracht 2}
Met een productietijd van gemiddeld 13.5 seconden voor iedere machine
werkt het systeem optimaal (onder de voorwaarde dat alle machines
gelijke productietijden hebben). De productie in 48 uur is ongeveer
64000 producten. 
\section{Opdracht 3}
Machine 1: 8 seconden gemiddeld

Machine 2: 8.5 seconden gemiddeld

Machine 3: 9.5 seconden gemiddeld

Machine 4: 17 seconden gemiddeld

Machine 5: 16.5 seconden gemiddeld 

Het resultaat is dat er per 48 uur ongeveer 80000 producten geproduceerd
worden. Een productiestijging van 25\%! Dus ja, door de machines in
productietijd te laten variëren is het mogelijk de productie nog verder
te verhogen zonder het systeem te laten vastlopen.
\section{Opdracht 4}
Verhoging van de snelheid van de rollenbanen zou inderdaad effect
hebben. Het opstapelende effect van de langzame opzetting bij de derde
en vierde machine zou zo sneller teniet gedaan kunne worden. Bij
versnelling van  de rollenbanen kunnen er meer dozen langs een opzetpunt
verplaatst worden voordat er gewacht moet worden voor een product dat
van de zijkant instroomt.