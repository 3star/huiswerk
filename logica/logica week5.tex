\documentclass{artikel1}
\usepackage[dutch]{babel}
\author{Paul Sohier 0806122}
\title{Logica hoofdstuk 4}


\begin{document}
\maketitle

\section{Opdracht 1}
$\exists m$ Sommige mensen\\
$\forall m$ Alle mensen\\
$\neg\exists m$ Er is niemand\\
$\neg\forall m$ Niet alle mensen\\
\section{Opdracht 2}
%∃u∀v∃w∀x∃y∀z¬P(u,v,w,x,y,z)
\begin{itemize}
\item[a] $\exists u \forall v \exists w \forall x \exists y \forall z \neg P (u,v,w,x,y,z)$
\item[b] Er zijn geen studenten die werken, alle studenten werken niet
\item[c] Niet alle studenten werken
\item[d] Er zijn geen studenten die zowel werken als studeren
\end{itemize}
\section{Opdracht 3}
\begin{itemize}
\item c impliceert a
\item b impliceert d
\item b en c spreken elkaar tegen
\item b en a zijn tegenstrijdig
\item c en d zijn tegenstrijdig
\end{itemize}
\section{Opdracht 4}
\begin{itemize}
\item[a]$man(x) \wedge kind(x,y)$
\item[b]$vrouw(x) \wedge kind(x,y)$
\item[c]$man(x) \wedge kind(x,z) \wedge kind(y,z) \wedge \neg gelijk(x,y)$ (Aangenomen dat een half-broer ook nog steeds als broer telt)
\item[d]$kind(y,z) \wedge kind(z,x) \wedge vrouw(x)$
\end{itemize}
\section{Opdracht 6}
\begin{itemize}
\item[a] Er is geen x die van alle y houdt.
\item[b] Alle x die van y houden, houden van z.
\item[c] Er is een x die van y houdt en van een z houdt.
\end{itemize}
\end{document}
